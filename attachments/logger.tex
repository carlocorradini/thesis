\chapter{Logging}
\label{cha:logging}

Logging is the process of preserving a record of events that occur in a computer
system, such as issues, errors, or just information on current operations. These
events might happen in the operating system or other applications. For each such
event, a message or log entry is recorded. These log messages may subsequently
be used to monitor and understand the system's operation, troubleshoot problems,
or during an audit. Logging is very crucial in multi-user applications to provide
a centralized view of the system's functionality\cite{logging}. \\ %
It is critical in the architecture implementation to log the numerous operations
that occur in each component of the cluster. The latter not only assists in understanding
how the general architecture performs and its status, but also the potential
causes of an incident. During the cluster's development, if any components
failed or did not behave as they should (e.g., no autonomous upscaling or downscaling),
monitoring and analyzing the numerous log files created was the only method to solve
the problems and understanding what and why happened. Furthermore, a logging system
is available for the multitude of utility scripts and applications that are not directly
employed with the cluster but rather during its general development. \\ %
Log interpretation is a challenging process. Log management system collects data
from a variety of sources with varying forms, purposes, and granularities. Maintaining
defined log levels helps give each entry meaning and better understand the significance
of the related log message\cite{logging_levels}. Section \ref{sec:logging_levels}
describes common logging levels that are also employed in cluster development and
implementation.

\section{Levels}
\label{sec:logging_levels}

A log level is a piece of information that indicates the importance of a log
message. It is a basic yet effective method of identifying log events from one another.
Furthermore, it is utilized to filter crucial information regarding the system state
to only those that are solely informational\cite{logging_levels}. \\ %
During the initialization phase, a level is chosen from the list below (sorted
in descending order of importance), with the expectation that the lower levels
are ignored in favor of the higher ones. The default logging level in various implementations
is often set to \texttt{INFO} level.

\begin{etaremune}
  \item The \texttt{DISABLED} has the highest possible rank and is intended to
  turn off logging.

  \item The \texttt{FATAL} level designates very severe error events that will presumably
  lead the application to abort.

  \item The \texttt{ERROR} level designates error events that might still allow
  the application to continue running.

  \item The \texttt{WARN} level designates potentially harmful situations.

  \item The \texttt{INFO} level designates informational messages that highlight
  the progress of the application at coarse-grained level.

  \item The \texttt{DEBUG} level designates fine-grained informational events that
  are most useful to debug an application.

  \item The \texttt{TRACE} level designates fine-grained informational events
  than the \texttt{DEBUG}.
\end{etaremune}

In table \ref{tbl:logging_levels}, is shown the correlation between the logging
level and the related logging message.

\begin{xltabular}
  {\textwidth} { >{\ttfamily}c | >{\large\ttfamily}c | >{\large\ttfamily}c | >{\large\ttfamily}c | >{\large\ttfamily}c | >{\large\ttfamily}c | >{\large\ttfamily}c }

  \multicolumn{1}{
  c
  |}{\backslashbox{\large{\textbf{Logging Level}}}{\large{\textbf{Logging Message}}}}
  & FATAL & ERROR & WARN & INFO & DEBUG & TRACE \\ \hhline{=======}

  SILENT & \cellcolor{bulmaRed} & \cellcolor{bulmaRed} & \cellcolor{bulmaRed} & \cellcolor{bulmaRed}
  & \cellcolor{bulmaRed} &\cellcolor{bulmaRed} \\ \hline

  FATAL & \cellcolor{bulmaGreen} & \cellcolor{bulmaRed} & \cellcolor{bulmaRed} &
  \cellcolor{bulmaRed} & \cellcolor{bulmaRed} & \cellcolor{bulmaRed} \\ \hline

  ERROR & \cellcolor{bulmaGreen} & \cellcolor{bulmaGreen} & \cellcolor{bulmaRed}
  & \cellcolor{bulmaRed} & \cellcolor{bulmaRed} & \cellcolor{bulmaRed} \\ \hline

  WARN & \cellcolor{bulmaGreen} & \cellcolor{bulmaGreen} & \cellcolor{bulmaGreen}
  & \cellcolor{bulmaRed} & \cellcolor{bulmaRed} & \cellcolor{bulmaRed}\\ \hline

  INFO & \cellcolor{bulmaGreen} & \cellcolor{bulmaGreen} & \cellcolor{bulmaGreen}
  & \cellcolor{bulmaGreen} & \cellcolor{bulmaRed} & \cellcolor{bulmaRed} \\
  \hline

  DEBUG & \cellcolor{bulmaGreen} & \cellcolor{bulmaGreen} & \cellcolor{bulmaGreen}
  & \cellcolor{bulmaGreen} & \cellcolor{bulmaGreen} & \cellcolor{bulmaRed} \\ \hline

  TRACE & \cellcolor{bulmaGreen} & \cellcolor{bulmaGreen} & \cellcolor{bulmaGreen}
  & \cellcolor{bulmaGreen} & \cellcolor{bulmaGreen} & \cellcolor{bulmaGreen}\\ \hline

  \caption{
  \centering
  Correlation between logging level and logging message
  \newline
  \begin{tabular}{l}
    \fcolorbox{black}{bulmaGreen}{\rule{0pt}{6pt}\rule{6pt}{0pt}}\quad Logging message is recorded and saved \\
    \fcolorbox{black}{bulmaRed}{\rule{0pt}{6pt}\rule{6pt}{0pt}}\quad Logging message is completely ignored   \\
  \end{tabular}}
  \label{tbl:logging_levels}
\end{xltabular}