\chapter{Projects}
\label{cha:projects}

During the development, I noticed that several portions of the code might be
turned into independent libraries and utility scripts that may be valuable to
other programmers as well as my single use-case scenario. Even though it is now external
to reCluster, this software is still an essential part of it, which is now open to
the entire community. \\ Shortly after the publication, I started receiving various
feedback and contributions, but most importantly, appreciation (GitHub stars\footnote{\url{https://docs.github.com/en/get-started/exploring-projects-on-github/saving-repositories-with-stars}}):
\begin{displayquote}
  Starring a repository shows appreciation to the repository maintainer for their
  work\cite{github_stars}
\end{displayquote}
% TODO Reference Philosophy and MIT
The three projects derived from the development of reCluster are briefly
illustrated and explained in the sections that follow. As stated in the Philosophy
section, everything is completely Open Source and available under the MIT
license.

\section{Node Exporter Installer}
\label{sec:projects_node_exporter_installer}

Available at \url{https://github.com/carlocorradini/node\_exporter\_installer} \\
% TODO Node exporter reference
% TODO K3s reference
Node exporter does not provide any installation script and the default procedure
(see \url{https://github.com/prometheus/node_exporter#installation-and-usage}) is
far from user-friendly and easily configurable. \\ Inspired by K3s \texttt{install.sh}\footnote{\url{https://github.com/k3s-io/k3s/blob/master/install.sh}}
script, Node exporter installer helps the user by automatically installing Node
exporter on the machine. Condensed in a single \texttt{install.sh} POSIX script,
it is easily configurable (see section
\ref{subsec:projects_node_exporter_installer_configuration}) and can be downloaded
and piped directly to \texttt{sh} in a single line.

\subsection{Example}
\label{subsec:projects_node_exporter_installer_example}

% TODO

\subsection{Configuration}
\label{subsec:projects_node_exporter_installer_configuration}

\begin{tabularx}
  {\textwidth} { l | X | c }

  \multicolumn{1}{ c |}{\large{\textbf{Name}}} &
  \multicolumn{1}{ c |}{\large{\textbf{Description}}} &
  \multicolumn{1}{ c }{\large{\textbf{Default Value}}} \\ \hline \hline

  \texttt{INSTALL\_NODE\_EXPORTER\_SKIP\_DOWNLOAD} & Skip downloading Node
  exporter.
  \newline
  A local executable binary must already exist at \texttt{<BIN\_DIR>/node\_exporter}
  \newline
  Useful in an Air-Gapped environment. % TODO Air-Gap environment reference
  & \texttt{false} \\ \hline

  \texttt{INSTALL\_NODE\_EXPORTER\_FORCE\_RESTART} & Force restarting Node exporter
  service. & \texttt{false} \\ \hline

  \texttt{INSTALL\_NODE\_EXPORTER\_SKIP\_ENABLE} & Do not enable Node exporter service
  at startup. & \texttt{false}\\ \hline

  \texttt{INSTALL\_NODE\_EXPORTER\_SKIP\_START} & Do not start Node exporter
  service. & \texttt{false} \\ \hline

  \texttt{INSTALL\_NODE\_EXPORTER\_SKIP\_FIREWALL} & Do not apply any firewall
  rules.
  \newline
  Supported firewalls are: \texttt{iptables}\footnote{\url{https://www.netfilter.org/projects/iptables}},
  \texttt{firewalld}\footnote{\url{https://firewalld.org}} and \texttt{UFW}\footnote{\url{https://wiki.ubuntu.com/UncomplicatedFirewall}}.
  & \texttt{false} \\ \hline

  \texttt{INSTALL\_NODE\_EXPORTER\_SKIP\_SELINUX} & Do not change \texttt{SELinux\footnote{\url{https://selinuxproject.org}}}
  context for Node exporter binary. & \texttt{false} \\ \hline

  \texttt{INSTALL\_NODE\_EXPORTER\_VERSION} & Node exporter version to download.
  & \texttt{latest} \\ \hline

  \texttt{INSTALL\_NODE\_EXPORTER\_BIN\_DIR} & Directory where to install Node exporter
  binary and uninstall script. & \makecell[Xt]{\centering{\texttt{/usr/local/bin} \\ or \\ \texttt{/opt/bin}}}
  \\ \hline

  \texttt{INSTALL\_NODE\_EXPORTER\_SYSTEMD\_DIR} & Directory where to install
  Systemd service files. & \texttt{/etc/systemd/system} \\ \hline

  \texttt{INSTALL\_NODE\_EXPORTER\_EXEC} & Node exporter arguments.
  \newline
  For a complete list of supported arguments see
  \url{https://github.com/prometheus/node\_exporter\#collectors} & \\
\end{tabularx}

\section{Inline}
\label{sec:projects_inline}

Available at \url{https://github.com/carlocorradini/inline}

\section{GraphQL Auth Directive}
\label{sec:projects_graphql_auth_diretive}

Available at \url{https://github.com/carlocorradini/graphql-auth-directive}