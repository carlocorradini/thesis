\chapter{Corollary Projects}
\label{cha:corollary_projects}

During the development, I noticed that several portions of the code might be
turned into independent libraries and utility scripts that may be valuable to
other programmers as well as my single use-case scenario. Even though it is now external
to reCluster, this software is still an essential part of it, which is now open to
the entire community. \\ %
Shortly after the publication, I began receiving some feedback, contributions,
and, most importantly, appreciation in the form of GitHub stars\footnote{\url{https://docs.github.com/en/get-started/exploring-projects-on-github/saving-repositories-with-stars}}.
\\ %
% TODO Reference Philosophy and MIT
The three projects derived from the development of reCluster are briefly
illustrated and explained in the sections that follow. As stated in the Philosophy
section, everything is completely Open Source and available under the MIT
license.

\section{Node Exporter Installer}
\label{sec:corollary_projects_node_exporter_installer}

% TODO Node exporter reference
% TODO K3s reference
Available at \url{https://github.com/carlocorradini/node\_exporter\_installer} \\ %
Used while installing reCluster on a Node. \\ %
Node exporter does not provide any installation script and the default procedure
(see \url{https://github.com/prometheus/node_exporter#installation-and-usage})
is far from user-friendly and easily configurable. \\ %
Inspired by K3s \texttt{install.sh}\footnote{\url{https://github.com/k3s-io/k3s/blob/master/install.sh}}
script, Node exporter installer helps the user by automatically installing Node
exporter on the machine. Condensed in a single \texttt{install.sh} POSIX script,
it is easily configurable (see section
\ref{subsec:corollary_projects_node_exporter_installer_configuration}) and can be
downloaded and executed directly by \texttt{sh} (see example \ref{subsubsec:corollary_projects_node_exporter_installer_example_basic}).

\subsection{Configuration}
\label{subsec:corollary_projects_node_exporter_installer_configuration}

Node exporter installer accepts environment variables only as configuration parameters.
This is done to avoid any potential conflict with the default behavior of Node
exporter, which employs argument flags. See
\url{https://github.com/prometheus/node\_exporter\#collectors} for a
comprehensive list of Node exporter configuration parameters. \\ %
The configuration settings allowed by Node exporter installer are shown in the
table below.

\begin{xltabular}
  {\textwidth} { >{\ttfamily}l | X | >{\ttfamily}c }

  \multicolumn{1}{ c |}{\large{\textbf{Name}}} &
  \multicolumn{1}{ c |}{\large{\textbf{Description}}} &
  \multicolumn{1}{ c }{\large{\textbf{Default Value}}} \\ \hline \hline

  INSTALL\_NODE\_EXPORTER\_SKIP\_DOWNLOAD & Skip downloading Node exporter.
  \newline
  A local executable binary must already exist at \texttt{<BIN\_DIR>/node\_exporter}
  \newline
  Useful in an Air-Gapped environment. % TODO Air-Gap environment reference
  & false \\ \hline

  INSTALL\_NODE\_EXPORTER\_FORCE\_RESTART & Force restarting Node exporter
  service. & false \\ \hline

  INSTALL\_NODE\_EXPORTER\_SKIP\_ENABLE & Do not enable Node exporter service at
  startup. & false \\ \hline

  INSTALL\_NODE\_EXPORTER\_SKIP\_START & Do not start Node exporter service. & false
  \\ \hline

  INSTALL\_NODE\_EXPORTER\_SKIP\_FIREWALL & Do not apply any firewall rules.
  \newline
  Supported firewalls are: \texttt{iptables}\footnote{\url{https://www.netfilter.org/projects/iptables}},
  \texttt{firewalld}\footnote{\url{https://firewalld.org}} and \texttt{UFW}\footnote{\url{https://wiki.ubuntu.com/UncomplicatedFirewall}}.
  & false \\ \hline

  INSTALL\_NODE\_EXPORTER\_SKIP\_SELINUX & Do not change \texttt{SELinux\footnote{\url{https://selinuxproject.org}}}
  context for Node exporter binary. & false \\ \hline

  INSTALL\_NODE\_EXPORTER\_VERSION & Node exporter version to download. & latest
  \\ \hline

  INSTALL\_NODE\_EXPORTER\_BIN\_DIR & Directory where to install Node exporter binary
  and uninstall script. & \makecell[Xt]{\centering{/usr/local/bin \\ or \\ /opt/bin}}
  \\ \hline

  INSTALL\_NODE\_EXPORTER\_SYSTEMD\_DIR & Directory where to install Systemd service
  files. & /etc/systemd/system \\ \hline

  INSTALL\_NODE\_EXPORTER\_EXEC & Node exporter configuration flags. & \\

  \caption{Node exporter installer configuration parameters}
\end{xltabular}

\subsection{Example}
\label{subsec:corollary_projects_node_exporter_installer_example}

This section shows some examples of how to use Node exporter installer. In both,
we use \texttt{curl} (a command-line tool for getting or sending data using URL
syntax\cite{curl}) to download the script from
\url{https://raw.githubusercontent.com/carlocorradini/node_exporter_installer/main/install.sh}
and pipe (redirect the output of a program to the input of another program) the
result to \texttt{sh} (UNIX shell command-line interpreter\footnote{\url{https://wikipedia.org/wiki/Unix_shell}})
for execution.

\subsubsection{Basic}
\label{subsubsec:corollary_projects_node_exporter_installer_example_basic}

Install Node exporter using the default configuration parameters for both
installer and binary.

\begin{lstlisting}[language=sh, morekeywords={curl, sh}, morestring={[s]{.sh}}, xleftmargin=\parindent, caption=Basic installation with default configuration parameters]
  curl \
    --silent \      # Do not show progress meter or error messages
    --show-error \  # Show an error message if it fails
    --fail \        # Fail fast with no output at all on server errors
    --location \    # If the requested resource has moved, redo the request to the new location
    "https://raw.githubusercontent.com/carlocorradini/node_exporter_installer/main/install.sh" \
    | sh -
\end{lstlisting}

\subsubsection{Advanced}
\label{subsubsec:corollary_projects_node_exporter_installer_example_advanced}

Install Node exporter v1.5.0 without starting the service automatically. Disable
all default collectors, leaving only CPU and memory statistics enabled.

\begin{lstlisting}[language=sh, morekeywords={curl, sh}, xleftmargin=\parindent, caption=Advanced installation with custom configuration parameters]
  curl \
    ... \
    | INSTALL_NODE_EXPORTER_VERSION="v1.5.0" \   # Download and install Node exporter v1.5.0
      INSTALL_NODE_EXPORTER_SKIP_START="true" \  # Do not start Node exporter service
      sh - \
        --collector.disable-defaults \             # Disable default collectors
        --collector.cpu \                          # Expose CPU statistics
        --collector.meminfo                        # Expose memory statistics
\end{lstlisting}

\section{Inline}
\label{sec:corollary_projects_inline}

% TODO Bundle reference
Available at \url{https://github.com/carlocorradini/inline} \\ %
Used while generating the final release bundle for distribution. \\ %
It is useful to be able to split a large script into many files to make it easier
to work with while still being able to distribute it as a single script. This
program reads an input file and produces an output file with all of the sources
inlined. \\ %
The source command, \texttt{.\ <FILE>} for POSIX shells or \texttt{source <FILE>}
for non-POSIX shells (e.g. Bash\footnote{\url{https://www.gnu.org/software/bash}}),
read, and executes commands from the file specified in the current shell
environment. It is useful to load functions, variables, and configuration files into
the current shell context\cite{source}. \\ %
Inline is a static tool that does not execute the input script. As a result, it
cannot determine the value of a variable dynamically if it is used inside the
source command path (i.e. \texttt{source "\$DIR/path/to/script/sh"}). To avoid
this, inline requires a hint on where to find the specified file.

\subsection{Features}
\label{subsec:corollary_projects_inline_features}

Many helpful features of Inline are described below.

\begin{itemize}
  \item POSIX standard-compliant.

  \item Sourcing with quotes, spaces, and more.

  \item Sourcing from global variable \texttt{\$PATH}.

  \item Sourcing from \texttt{ShellCheck} (a shell script static analysis tool\footnote{\url{https://www.shellcheck.net}})
    source directive\footnote{\url{https://www.shellcheck.net/wiki/Directive}}.
    \newline
    This is considered a hint to Inline and is used only if the source path is invalid.
    \newline
    \begin{lstlisting}[language=sh, morekeywords={., source}, numbers=none, aboveskip=0pt, belowskip=0pt, abovecaptionskip=0pt, belowcaptionskip=0pt]
      # shellcheck source=path/to/script.sh
      . "$DIR/path/to/script.sh"

      # shellcheck source=path/to/script.sh
      source "$DIR/path/to/script.sh"
    \end{lstlisting}

  \item Recursive sources. If a sourced file contains additional source commands,
    they are also inlined and included in the final output script.
    \newline
    It should be noted that these can cause infinite recursion.

  \item Recursion detection. To avoid infinite recursion, an exception is thrown
    if a file is sourced multiple times. This is accomplished through the use of
    an internal cache that saves the absolute path to each sourced file. If the path
    to a script file in a source command is already in the cache, a recursion is
    detected.

  \item Shebang removal in sourced files.
    \newline
    A shebang is the character sequence consisting of the characters number sign
    and exclamation mark (\texttt{\#!}) at the beginning of a script. In a Unix-like
    operating system, when a text file with a shebang is used as if it is an
    executable, the program loader executes the specified interpreter program,
    passing as an argument the path that was initially used when attempting to
    run the script, so that the program can use the file as input data\cite{shebang}
    \newline
    The shell interpreter only allows one shebang per script. As a result, the
    shebang is only allowed in the input script file, while it is automatically removed
    in all sourced script files.

  \item Avoid inlining a certain source file. If the inline skip comment
    directive (\texttt{\# inline skip}) is present before a source command, the
    latter is ignored and not inlined. As a consequence, the final output script
    includes the original command, unaltered and unaligned.
    \newline
    It is worth noting that the skip directive also works with ShellCheck. The sole
    requirement is that there are no blank lines between the directives and the source
    command.
    \newline
    \begin{lstlisting}[language=sh, morekeywords={., source}, numbers=none, aboveskip=0pt, belowskip=0pt, abovecaptionskip=0pt, belowcaptionskip=0pt]
    # inline skip
    # shellcheck source=path/to/script.sh
    . "$DIR/path/to/script.sh"

    # shellcheck source=path/to/script.sh
    # inline skip
    source "$DIR/path/to/script.sh"
  \end{lstlisting}
\end{itemize}

\subsection{Configuration}
\label{subsec:corollary_projects_inline_configuration}

Inline's behavior is easily customizable by using argument flags, which begin with
a double dash followed by the property name and an optional value (\texttt{--NAME
[VALUE]}). \\ %
The accepted configuration parameters are listed in the table below.

\begin{xltabular}
  {\textwidth} { >{\ttfamily}l | X | >{\ttfamily}c }

  \multicolumn{1}{ c |}{\large{\textbf{Name}}} &
  \multicolumn{1}{ c |}{\large{\textbf{Description}}} &
  \multicolumn{1}{ c }{\large{\textbf{Default Value}}} \\ \hline \hline

  --in-file <FILE> & Input script file (\texttt{<FILE>}).
  \newline
  If the file does not exist, an error is thrown. & \\ \hline

  --out-file <FILE> & Output script file (\texttt{<FILE>}).
  \newline
  If a file with the same name and path already exists, an error is thrown.
  Unless the \texttt{--overwrite} argument is given, the latter is always true (see
  option below).
  \newline
  If no value is specified, the name of the output file is determined by
  examining the original input file. The first portion is the original name,
  followed by a \texttt{.inlined} string, and finally, if present, the extension
  (usually \texttt{.sh}). & <NAME>.inlined[EXTENSION] \\ \hline

  --overwrite & Replace the input file.
  \newline
  The inlined result replaces the original file content. This is the same as
  setting the output file's value as the original input file, except that the existence
  check is bypassed (see option above). & false \\ \hline

  --log-level <LEVEL> & Logging level (\texttt{<LEVEL>}).
  \newline
  A level is chosen from the list below (in descending order of priority), with
  the expectation that lower levels are ignored in favor of the higher ones.
  \begin{itemize}[noitemsep]
    \item \texttt{fatal}
      \newline
      Extremely serious error events that most likely cause the program to
      terminate.

    \item \texttt{warn}
      \newline
      Potentially dangerous conditions.

    \item \texttt{info}
      \newline
      Informational messages that emphasize the application's progress at a
      coarse-grained level.

    \item \texttt{debug}
      \newline
      Fine-grained informative events are most helpful when debugging an
      application.

    \item \texttt{silent}
      \newline
      Disable logging.
  \end{itemize}
  & info \\ \hline

  --disable-color & Disable terminal colors (enabled by default). & false \\ \hline

  --help & Display a help message and terminate (successfully). & \\

  \caption{Inline configuration parameters}
\end{xltabular}

\subsection{Example}
\label{subsec:corollary_projects_inline_example}

This section provides an example of how Inline works and how it may be used. \\ %
There are two script files, both of which begin with a shebang. There is a print
command in the first file, \texttt{hello.sh}, that outputs \texttt{Hello} and an
empty space, followed by a source command that includes the script file \texttt{world.sh}.
The format of the second file, \texttt{world.sh}, is the same as the first, except
that the print command outputs \texttt{World!} and the newline character (\texttt{\char`\\n}),
and there are no source commands. \\ %
The following two listings show the contents of the two files:

\noindent
\hspace{.775\parindent}
\begin{minipage}[t]{.45\textwidth}
  \begin{lstlisting}[language=sh, morekeywords={., printf}, caption=Input script \texttt{hello.sh}]
    #!/usr/bin/env sh

    printf "Hello "

    . "world.sh"
  \end{lstlisting}
\end{minipage}
\hfill
\begin{minipage}[t]{.45\textwidth}
  \begin{lstlisting}[language=sh, morekeywords={., printf}, caption=Sourced script \texttt{world.sh}]
    #!/usr/bin/env sh

    printf "World!\n"
  \end{lstlisting}
\end{minipage}

The goal is to inline the script file \texttt{hello.sh} and produce an output
file that has no source commands. When the command %
\lstinline[language=sh, deletekeywords={in}, alsoletter={.}, morekeywords={inline.sh}]{./inline.sh --in-file "hello.sh"} %
is executed, the following result is obtained:

\begin{lstlisting}[language=sh, morekeywords={., printf}, xleftmargin=\parindent, caption=Inlined script \texttt{hello.inlined.sh}]
  #!/usr/bin/env sh

  printf "Hello "

  # . "world.sh"

  printf "World!\n"
\end{lstlisting}

Note the unique shebang as well as the source command that has been commented
out with the symbol \texttt{\#}. Furthermore, because no \texttt{--out-file}
option is used, the final output file is named \texttt{hello.inlined.sh}, which does
not override the original input file.

\section{GraphQL Auth Directive}
\label{sec:corollary_projects_graphql_auth_diretive}

% TODO GraphQL reference
% TODO Node.js reference
Available at \url{https://github.com/carlocorradini/graphql-auth-directive} \\ %
Used in the \texttt{GraphQL} API. \\ %
A custom \texttt{GraphQL} directive that protects resources from unauthenticated
and unauthorized access in high-security contexts. It is available in all major
\texttt{Node.js} registries as \texttt{graphql-auth-directive}. \\ %
A directive is an identifier preceded by a \texttt{@} character, optionally
followed by a list of named arguments, which can appear after almost any form of
syntax in the \texttt{GraphQL} query or schema languages\cite{graphql_directive}.
\\ %
The \texttt{GraphQL} context holds all important information about the current
request. A \texttt{GraphQL} context is an object that is shared by all resolvers
in a given execution. It helps store data such as authentication information,
the current user, database connections, data sources, and other information required
to operate the business logic\cite{graphql_context}. It is important to note that
the context does not have to follow a predefined structure; rather, it is highly
flexible to the user's implementation.

\subsection{Configuration}
\label{subsec:corollary_projects_graphql_auth_diretive_configuration}

\begin{xltabular}
  {\textwidth} { >{\ttfamily}l | X | >{\ttfamily}c }

  \multicolumn{1}{ c |}{\large{\textbf{Name}}} &
  \multicolumn{1}{ c |}{\large{\textbf{Description}}} &
  \multicolumn{1}{ c }{\large{\textbf{Default Value}}} \\ \hline \hline

  name & Directive name.
  \newline
  If a name different from the default is specified, it must be reflected in the
  schema where the directive is used, otherwise, an error is thrown. & auth \\
  \hline

  auth & A function or class that handles authentication and authorization.
  \newline
  The current context (which contains information about the current request) and
  the roles and permissions required by the requested resource must be accepted
  as arguments by the implementation. If access to the requested resource is granted,
  the boolean value \texttt{true} is returned; otherwise, \texttt{false} is
  returned if access is denied.
  \newline
  A default basic auth function is already implemented and checks for the existence
  of an authorized applicant and that its roles and permissions overlap with those
  of the requested resource. & \\ \hline

  authMode & Auth mode if access is not granted.
  \newline
  Methodology for informing the requestor that access to the resource has been
  denied. It is often desirable to hide the important information that the request
  is failing due to an auth check and instead deliver an error/informational response
  stating that the requested resource simply does not exist.
  \newline
  The first mode, \texttt{ERROR}, throws an authentication or authorization error
  (see below), whereas the second mode, \texttt{NULL}, returns the value \texttt{null}
  (empty). & ERROR \\ \hline

  roles & Roles configuration.
  \newline
  An object with two properties:
  \begin{itemize}
    \item \texttt{enumName}
      \newline
      Defines the array type, which is by default a \texttt{String}.
      \newline
      It is standard practice in \texttt{GraphQL} and in programming languages
      to map a set of values to an enum. An enumeration type is a special kind of
      scalar that is restricted to a particular set of allowed values\cite{graphql_enum}.
      \newline
      This option restricts the allowed values of roles to a specific set rather
      than a general string.

    \item \texttt{default}
      \newline
      Roles that are required by default.
      \newline
      No roles are required by default. As a result, access to a protected resource
      can only be provided with authentication. Overriding this option enables
      authorization by default, requiring the requestor to have at least one matching
      role.
  \end{itemize}
  & \makecell[Xt]{\centering{enumName: String \\ default: []}} \\ \hline

  permissions & Permissions configuration.
  \newline
  An object with the same properties as roles configuration (see above). & \makecell[Xt]{\centering{enumName: String \\ default: []}}
  \\ \hline

  authenticationError & Authentication error class. The error is thrown if there
  is no authenticated requestor in the current context. By default, a generic error
  is thrown. The class must extend the \texttt{Error} class. &
  AuthenticationError \\ \hline

  authorizationError & Authorization error class. The error is thrown if the
  roles and/or permissions of the requestor do not overlap with those of the
  requested resource. By default, a generic error is thrown. The class must extend
  the \texttt{Error} class. & AuthorizationError \\ \hline

  container & Dependency Injection Container.
  \newline
  Dependency injection is a design pattern that shifts the responsibility of resolving
  dependencies to a dedicated dependency injector that knows which dependent objects
  to inject into application code\cite{dependency_injection}.
  \newline
  It should be noted that dependency injection is only available if \texttt{auth}
  is a class type. & IOCContainer \\

  \caption{GraphQL auth directive configuration parameters}
\end{xltabular}

\subsection{Type Definition}
\label{subsec:corollary_projects_graphql_auth_diretive_type_definition}

\begin{lstlisting}[language=graphql, xleftmargin=\parindent, caption=GraphQL auth directive type definition]
  """
  Protect the resource from unauthenticated and unauthorized access.
  """
  directive @auth(
    """
    Allowed roles to access the resource.
    """
    roles: [String!]! = []
    """
    Allowed permissions to access the resource.
    """
    permissions: [String!]! = []
  ) on FIELD | FIELD_DEFINITION | OBJECT
\end{lstlisting}

\subsection{Example}
\label{subsec:corollary_projects_graphql_auth_diretive_example}

\begin{lstlisting}[language=graphql, xleftmargin=\parindent, caption=TODO]
  enum Role {
    ADMIN
  }

  enum Permission {
    VIEW
  }

  type Query {
    unprotected: String!
    protected: String! @auth
    secret: String! @auth(roles: [ADMIN], permissions: [VIEW])
  }
\end{lstlisting}