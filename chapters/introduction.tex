\chapter{Introduction}
\label{cha:introduction}

The purpose of this chapter is not to illustrate any cluster architecture or implementation,
but rather to provide an overview of the overall context, depicted in section \ref{sec:introduction_context},
and core principles, depicted in section \ref{sec:introduction_principles}, that
serve as foundations for the overall cluster design, both theoretical and implemented.
\\ %
These so-called foundations are not restricted to this project and/or environment,
but can be considered as the minimum requirements and principles to obtain a
greener, energy and resource-aware software in general that is not only for the present,
but also, and must be, for the future, because we, as the overall ICT community,
have mostly ignored them in the past without giving them the appropriate weight.
\\ %

reCluster is heterogeneous because it is composed of numerous and diverse
components, ranging from computers to networking equipment, that have been
reutilized after decommissioning. The fundamental goal of reCluster is to establish
a cluster (hence the name) without reinventing the wheel, but rather to increase
reusability by transforming and adapting heterogeneous hardware and software
components designed for similar environments, enhancing the overall reduction of
both energy consumption and resources involved without compromising overall performance,
responsiveness, and ease of use.

\section{Context}
\label{sec:introduction_context}

There is a need to identify the overall context that the project, design, and
architecture should and must address. A key aspect of the context is that it is not
static, but rather dynamic, and therefore may be decreased, expanded, and reshaped
to better meet the requirements of the organization in charge of the cluster while
adhering to the core principles. \\ %
A cluster is composed of several components, including hardware and software, that
operate together as a single entity to accomplish one homogenous goal, or
multiple and different goals. The hardware components, which are typically
computers of different kinds ranging from desktop computers to single board
computers, are connected in a single network: the cluster network, by additional
hardware components particularly designed for communication. Because multiple
applications can achieve the same result, are constantly evolving with newer features
and capabilities, and can be employed or not depending on the (current) needs
and requirements, the software components can vary and are as heterogeneous as the
hardware, if not more. Between hardware and software, two major distinctions
complement one another. The first distinction is that hardware, particularly
reconditioned hardware, is limited in its ability to be modified and transformed,
whereas software is essentially limitless in this regard and may achieve the
same goal using multiple technologies and/or languages. Second, there is no
distinction between the various hardware components in terms of whether they are
used for cluster operations or user operations because they are all employed for
cluster operations. Software components can be divided into low-level ones that are
used to keep the overall cluster operational and high-level ones that are used by
and for the users for their specific services. In this context, software in all of
its aspects must be used to compensate for the shortcomings of the hardware since
it can be customized to fill the gaps where the hardware lacks. \\ %
Most used hardware components may be reconditioned by reusing decommissioned components
from consumers and organizations due to upgrades. Because newer components are
often more performant and energy-efficient than older ones, the hardware is directly
and automatically managed by the various cluster-related software deployed. \\ %
The cluster must be as simple as possible, with human interaction minimized to
nearly zero, and automatic solutions preferred. The most essential automated
solution that must be implemented in a cluster is autoscaling, both upscaling and
downscaling. In upscaling, the cluster automatically bootstraps an inactive hardware
component. Whereas in downscaling, the cluster automatically terminates an active
hardware component. Selecting which component(s) to autoscale can follow multiple
criteria, such as choosing nodes that are more power efficient or performant, or
even a combination of the two. The latter is an excellent illustration of the
combination of hardware and software that needs to be present in the cluster. \\ %
The cluster should be used in a wide range of environments and use-case
scenarios, from a modest and local deployment at home to a large and remote deployment
comparable to that found in a data center. Given that the various software
manages them automatically using the autoscaling technique, there should be no limit
to the number of computers. \\ %
If there is no need to operate the cluster for specific periods and it is therefore
considered worthless, it can be completely shut down, with the benefit of reducing
its overall power consumption to zero. When it is necessary again, it is reactivated
and becomes operational. Although the latter may create some performance and/or
responsiveness issues, because no physical machines are performing any service,
it is far preferable to an always-on technique because the machine continues to
drain energy for no purpose. The latter impacts can be mitigated if the organization
can calculate different models and/or employs a single board computer as the always-on
hardware component because its overall power consumption is lower than a normal lamp
and substantially lower than a normal desktop computer. As stated previously, the
context is dynamic, and hence the latter is dependent on the organization in
charge of the cluster. \\ %

As a result, the fundamental focal point is the combination of all latter points,
and there may be others, which makes the cluster resemble less heterogeneous and
more of a distinct homogenous and automatic entity.

\section{Principles}
\label{sec:introduction_principles}

This section is dedicated to illustrating some of the fundamental principles around
which the overall design is based. These principles should (must) be seen as the
foundations for a more sustainable, resource-aware, free and open-source, and energy-efficient
architecture that may be utilized not only in this specific context, but in all
aspects of ICT. \\ %
It should be highlighted that these principles should not have a negative impact
on overall performance and responsiveness, but rather should be considered almost
as important, if not more, than performance and responsiveness. The latter two
currently constitute the sole and most significant characteristics to consider
when examining software, but in the future, it should be necessary, almost mandatory,
to include, maybe gradually but starting to consider them as parameters for comparisons
and essential aspects of a program. \\ %
The applicability of these principles is determined by the organization's requirements.
Yet, some of these principles should be investigated and implemented in the future
for greener and more resource-aware software. \\ %

The proposed list can undoubtedly be improved upon and/or expanded with
additional, and possibly more severe, principles. Nevertheless, since these are
the ones on which the current architecture is based, it can be adjusted and transformed
in the future. It is not only possible but also necessary for the software
industry to move toward a greener and more sustainable future.

\subsection{Sustainable Development Goals}
\label{subsec:introduction_principles_sustainable_development_goals}

\subsection{Insourcing}
\label{subsec:introduction_principles_insourcing}

% TODO Direct Control

% TODO Privacy

\subsection{Energy Impact}
\label{subsec:introduction_principles_energy_impact}

\subsection{Resource Awareness}
\label{subsec:introduction_principles_resource_awareness}

\subsection{Reusability}
\label{subsec:introduction_principles_reusability}

% TODO REPAIR: https://frame.work

% TODO Electronic Waste

\subsection{Standardization}
\label{subsec:introduction_principles_standardization}

% TODO Migration

\subsection{Dependencies Reduction}
\label{subsec:introduction_principles_dependencies_reduction}

\subsection{Free And Open-Source Software}
\label{subsec:introduction_principles_reusability_free_and_open_source_software}

\section{State Of The Art}
\label{sec:introduction_state_of_the_art}

% TODO Problems/Discussions
% TODO Responsiveness and SLA
% TODO Cloud Provider
% TODO SLA consumption

% TODO reCluster paper