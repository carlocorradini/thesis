\chapter{Implementation}
\label{cha:implementation}

\epigraph{Ideas are easy, Implementation is hard.}{Guy Kawasaki}

Most of the implementation details and general decisions made for developing a
real-world functional project, based on the design architecture presented in chapter
\ref{cha:architecture}, are discussed in this chapter. It is organized in such a
way that it starts with the foundation and then progresses to the autoscaling of
worker nodes before moving on to example deployments of various applications in the
cluster. \\ %
The technology and programming languages utilized in the implementation are heterogeneous.
Nevertheless, according to a shared set of APIs and pre-defined structures, they
interact as a distinct and homogenous entity that maintains the entire cluster
in an active and healthy state. \\ %
Furthermore, several of the components and/or technologies discussed in this chapter
are interchangeable with other solutions. The latter is highly valuable for
organizations since it provides more configuration freedom and final control
over the cluster while continuously reflecting its ultimate goal.

\section{Distributions}
\label{sec:implementation_distributions}

% TODO Reference introduction dependencies

The foundation for almost every software relies on the need of a solid and
consistent Operating System (OS).

As written in section \ref{subsec:architecture_components_node}, the OS needs to
be installed on every node in the cluster and must be based on the \texttt{Linux
Kernel}. The latter is a fundamental requirement since most of the technologies used
in the implementation relies on Linux primitives and core functionalities to be able
to work flawlessly. As an example, all Kubernetes distributions requires the \texttt{cgroup
v2}\footnote{\url{https://www.kernel.org/doc/html/latest/admin-guide/cgroup-v2.html}}
kernel module to enforce resource management for pods and containers, which includes
cpu/memory requests and limits for containerized workloads\footnote{\url{https://kubernetes.io/docs/concepts/architecture/cgroups}}.

The section is named Distributions and not Operating System to reflect the possibility
of the implementation architecture to be compatible with multiple Linux distributions.
A Linux distribution, often shortened to Linux distro, is an operating system
compiled from components developed by various open source projects and
programmers. Each distribution includes the Linux kernel (foundation of the
operating system), the GNU shell utilities (terminal interface and commands), the
X server (for a graphical desktop), the desktop environment, a package management
system, an installer and other services. Many components are developed independently
from each other and are distributed in source code form. A single Linux
distribution may contain thousands of software packages, utilities and applications.
Linux distributions compile code from open source projects and combine it into a
single operating system that can be installed and booted up. Linux distributions
are available for a multitude of different use-case, such as desktop computers, servers
without a graphical interface, super computers, mobile devices, and more\cite{https://www.suse.com/suse-defines/definition/linux-distribution}.

A Linux Distribution, to be fully compatible with the implementation
architecture, needs to satisfies three main requirements. They are described in
the list below: 1. The version of the Linux Kernel must be compatible with the chosen
Kubernetes version. To support newer functionalities and/or enhancing overall
security, newer versions of K8s may require newer versions of the Linux Kernel.
2. All required packages, described and listed in section \ref{subsec:implementation_distributions_packages},
must be already installed and globally available in the system. Before the execution
of the installation program, described in section
\ref{sec:implementation_installer}, a check on every required packages is performed.
If even a single package is missing from the current system, the installation fails
with an error message describing what packages are missing. 3. A supported and
compatible Init System, described in section \ref{subsec:implementation_distributions_init_system}.
The implementation does supports OpenRC and Systemd. Currently, they are the two
most widespread init sistems and most of the available distributions are based
on one of them.

Finally, two custom Linux Distributions, shown in section
\ref{subsec:implementation_distributions_iso_image}, have been already prepared and
configured. This is done to simplify the overall configuration, removing possible
incompatibilities while also enhancing implementation and testing speed of the cluster.
Both of them are based on preexisting Linux Distribution that are known to be
stable, easy to configure and require minimal resources to work. Moreover, they are
based on different init systems forcing the overall implementation to be compatible
and tested with both of them.

\subsection{Packages}
\label{subsec:implementation_distributions_packages}

\subsection{Init System}
\label{subsec:implementation_distributions_init_system}

\subsubsection{OpenRC}
\label{subsec:implementation_distributions_init_system_openrc}

\subsubsection{Systemd}
\label{subsec:implementation_distributions_init_system_systemd}

\begin{wrapfigure}
  {r}{.25\textwidth}
  \centering
  \includegraphics[width=.25\textwidth]{images/logos/systemd.png}
  \caption{Systemd logo}
\end{wrapfigure}

\subsection{ISO image}
\label{subsec:implementation_distributions_iso_image}

\subsubsection{Generation}
\label{subsubsec:implementation_distributions_iso_generation}

\subsubsection{Alpine Linux}
\label{subsubsec:implementation_distributions_iso_alpine_linux}

\begin{wrapfigure}
  {r}{.25\textwidth}
  \centering
  \includegraphics[width=.25\textwidth]{images/logos/alpine.png}
  \caption{Alpine Linux logo}
\end{wrapfigure}

Alpine Linux\footnote{\url{https://www.alpinelinux.org}} is a lightweight, security-focused,
and resource-efficient distribution. In recent years, has grown in popularity as
the foundation for the majority of Docker images. It is based on musl\footnote{\url{https://musl.libc.org}}
(an implementation of the C standard library, an alternative to glibc), BusyBox\footnote{\url{https://busybox.net}}
(a single, small executable that combines tiny versions of many common UNIX utilities),
a custom package manager called APK\footnote{\url{https://wiki.alpinelinux.org/wiki/Alpine_Package_Keeper}}
and the OpenRC init system ()\cite{alpine_linux}. \\ % TODO OpenRC reference
Alpine Linux is the preferred reCluster distribution. It has been utilized and
tested during development and is now present in every cluster node.

\subsubsection{Arch Linux}
\label{subsubsec:implementation_distributions_iso_arch_linux}

\begin{wrapfigure}
  {r}{.25\textwidth}
  \centering
  \includegraphics[width=.25\textwidth]{images/logos/arch.png}
  \caption{Arch Linux logo}
\end{wrapfigure}

Arch Linux\footnote{\url{https://archlinux.org}} is a general-purpose
distribution that focuses on simplicity, minimalism, and code elegance. It is based
on Systemd init system () % TODO Systemd reference
and strives to stay bleeding edge, offering the latest stable versions of most
software through Pacman\footnote{\url{https://wiki.archlinux.org/title/Pacman}}
package manager. Uses a rolling release system that allows one-time installation
and perpetual software upgrades, allowing to not reinstall or upgrade the system
from one version to the next. Notably, Arch Linux is the foundation for several
many popular enterprise-grade distributions, such as Manjaro\footnote{\url{https://manjaro.org}}
and SteamOS\footnote{\url{https://store.steampowered.com/steamos}},
demonstrating its customizability, power, and stability\cite{arch_linux}.

\section{Dependencies}
\label{sec:implementation_dependencies}

% TODO Dependencies presentation

\subsection{Air-Gap Environment}
\label{subsec:implementation_dependencies_air_gap_environment}

\subsection{Management}
\label{subsec:implementation_dependencies_management}

\subsubsection{Configuration}
\label{subsec:implementation_dependencies_management_configuration}

\subsubsection{Script}
\label{subsec:implementation_dependencies_management_script}

\subsection{K3s}
\label{subsec:implementation_dependencies_k3s}

\begin{wrapfigure}
  {r}{.25\textwidth}
  \centering
  \includegraphics[width=.25\textwidth]{images/logos/k3s.png}
  \caption{K3s logo}
\end{wrapfigure}

\subsection{Node Exporter}
\label{subsec:implementation_dependencies_node_exporter}

\subsection{Prometheus}
\label{subsec:implementation_dependencies_prometheus}

\begin{wrapfigure}
  {r}{.25\textwidth}
  \centering
  \includegraphics[width=.25\textwidth]{images/logos/prometheus.png}
  \caption{Prometheus logo}
\end{wrapfigure}

\subsection{Autoscaler}
\label{subsec:implementation_dependencies_autoscaler}

\subsubsection{Cluster Autoscaler}
\label{subsec:implementation_dependencies_autoscaler_cluster_autoscaler}

\section{Installer}
\label{sec:implementation_installer}

\subsection{POSIX Shell}
\label{subsec:implementation_installer_posix_shell}

\subsection{Configuration Arguments}
\label{subsec:implementation_installer_configuration_arguments}

\subsection{Configuration Files}
\label{subsec:implementation_installer_configuration_files}

\subsubsection{Certificates}
\label{subsubsec:implementation_installer_configuration_files_certificates}

\subsubsection{K3s}
\label{subsubsec:implementation_installer_configuration_files_k3s}

\subsubsection{K8s}
\label{subsubsec:implementation_installer_configuration_filesn_k8s}

\subsubsection{Node exporter}
\label{subsubsec:implementation_installer_configuration_files_node_exporter}

\subsubsection{reCluster}
\label{subsubsec:implementation_installer_configuration_files_recluster}

\subsubsection{SSH}
\label{subsubsec:implementation_installer_configuration_files_ssh}

\subsection{Node Facts}
\label{subsec:implementation_installer_node_facts}

\subsection{Node Information}
\label{subsec:implementation_installer_node_information}

\subsection{Node Benchmarks}
\label{subsec:implementation_installer_node_benchmarks}

\subsubsection{SysBench}
\label{subsubsec:implementation_installer_node_benchmarks_sysbench}

\subsection{Node Power Consumption}
\label{subsec:implementation_installer_node_power_consumption}

\subsubsection{CloudFree Smart Plug}
\label{subsubsec:implementation_installer_node_power_consumption_cloudfree_smart_plug}

\subsection{Node Registration}
\label{subsec:implementation_installer_node_registration}

\subsubsection{K8s Node Label And Name}
\label{subsubsec:implementation_installer_node_registration_k8s_node_label_and_name}

\subsection{Cluster Initialization}
\label{subsec:implementation_installer_cluster_initialization}

\subsubsection{Kubeconfig}
\label{subsubsec:implementation_installer_cluster_initialization_kubeconfig}

\subsubsection{Database}
\label{subsubsec:implementation_installer_cluster_initialization_database}

\subsubsection{Server}
\label{subsubsec:implementation_installer_cluster_initialization_server}

\subsubsection{Admin And Autoscaler Users}
\label{subsubsec:implementation_installer_cluster_initialization_admin_and_autoscaler_users}

\subsubsection{K8s}
\label{subsubsec:implementation_installer_cluster_initialization_k8s}

\subsection{Services}
\label{subsec:implementation_installer_services}

\subsubsection{Start}
\label{subsubsec:implementation_installer_services_start}

\subsubsection{Stop}
\label{subsubsec:implementation_installer_services_stop}

\subsubsection{Boot}
\label{subsubsec:implementation_installer_services_boot}

\subsubsection{Shutdown}
\label{subsubsec:implementation_installer_services_shutdown}

\section{Server}
\label{sec:implementation_server}

\subsection{Database}
\label{subsec:implementation_server_database}

\subsubsection{PostgreSQL}
\label{subsubsec:implementation_server_database_postgresql}

\begin{wrapfigure}
  {r}{.25\textwidth}
  \centering
  \includegraphics[width=.25\textwidth]{images/logos/postgresql.png}
  \caption{PostgreSQL logo}
\end{wrapfigure}

\subsubsection{Prisma}
\label{subsubsec:implementation_server_database_prisma}

\begin{wrapfigure}
  {r}{.25\textwidth}
  \centering
  \includegraphics[width=.25\textwidth]{images/logos/prisma.png}
  \caption{Prisma logo}
\end{wrapfigure}

\subsubsection{Schema}
\label{subsubsec:implementation_server_database_schema}

\subsection{GraphQL}
\label{subsec:implementation_server_graphql}

\subsubsection{GraphQL vs REST}
\label{subsubsec:implementation_server_graphql_graphql_vs_rest}

\subsubsection{Schema}
\label{subsubsec:implementation_server_graphql_schema}

\subsection{Node Pool}
\label{subsec:implementation_server_node_pool}

\subsection{Node Registration}
\label{subsec:implementation_server_node_registration}

\subsubsection{JSON Web Token}
\label{subsubsec:implementation_server_node_registration_json_web_token}

\subsection{Upscaling}
\label{subsec:implementation_server_upscaling}

\subsubsection{Wake-on-LAN}
\label{subsubsec:implementation_server_scale_up_wake_on_lan}

\subsection{Downscaling}
\label{subsec:implementation_server_downscaling}

\subsubsection{SSH}
\label{subsubsec:implementation_server_scale_up_ssh}

\section{Autoscaling}
\label{sec:implementation_autoscaling}

\subsection{Vertical Pod Autoscaler}
\label{subsec:implementation_autoscaling_vertical_pod_autoscaler}

\subsection{Horizontal Pod Autoscaler}
\label{subsec:implementation_autoscaling_horizontal_pod_autoscaler}

\subsection{Cluster Autoscaler}
\label{subsec:implementation_autoscaling_cluster_autoscaler}

\subsubsection{Cloud Provider}
\label{subsubsec:implementation_autoscaling_cluster_autoscaler_cloud_provider}

\subsubsection{Configuration}
\label{subsubsec:implementation_autoscaling_cluster_autoscaler_configuration}

% TODO Cloud Configuration

\subsubsection{Upscaling}
\label{subsubsec:implementation_autoscaling_cluster_autoscaler_upscaling}

\subsubsection{Downscaling}
\label{subsubsec:implementation_autoscaling_cluster_autoscaler_downscaling}