\chapter{Implementation}
\label{cha:implementation}

\epigraph{Ideas are easy, Implementation is hard.}{Guy Kawasaki}

\section{Server}
\label{sec:implementation_server}

\subsection{Database}
\label{subsec:implementation_server_database}

\subsubsection{PostgreSQL}
\label{subsubsec:implementation_server_database_postgresql}

\begin{wrapfigure}
  {r}{.125\textwidth}
  \centering
  \includegraphics[width=.125\textwidth]{images/logos/postgresql.png}
\end{wrapfigure}

\subsubsection{Prisma}
\label{subsubsec:implementation_server_database_prisma}

\begin{wrapfigure}
  {r}{.125\textwidth}
  \centering
  \includegraphics[width=.125\textwidth]{images/logos/prisma.png}
\end{wrapfigure}

\subsubsection{Schema}
\label{subsubsec:implementation_server_database_schema}

\subsection{GraphQL}
\label{subsec:implementation_server_graphql}

\subsubsection{GraphQL vs REST}
\label{subsubsec:implementation_server_graphql_graphql_vs_rest}

\subsubsection{Schema}
\label{subsubsec:implementation_server_graphql_schema}

\subsection{Node Pool}
\label{subsec:implementation_server_node_pool}

\subsection{Node Registration}
\label{subsec:implementation_server_node_registration}

\subsubsection{JSON Web Token}
\label{subsubsec:implementation_server_node_registration_json_web_token}

\subsection{Scale Up}
\label{subsec:implementation_server_scale_up}

\subsubsection{Wake-on-LAN}
\label{subsubsec:implementation_server_scale_up_wake_on_lan}

\subsection{Scale Down}
\label{subsec:implementation_server_scale_down}

\subsubsection{SSH}
\label{subsubsec:implementation_server_scale_up_ssh}

\section{Distributions}
\label{sec:implementation_distributions}

\subsection{Packages}
\label{subsec:implementation_distributions_packages}

\subsection{Init System}
\label{subsec:implementation_distributions_init_system}

\subsubsection{OpenRC}
\label{subsec:implementation_distributions_init_system_openrc}

\subsubsection{Systemd}
\label{subsec:implementation_distributions_init_system_systemd}

\subsection{ISO}
\label{subsec:implementation_distributions_iso}

\subsubsection{Generation}
\label{subsubsec:implementation_distributions_iso_generation}

\subsubsection{Alpine Linux}
\label{subsubsec:implementation_distributions_iso_alpine_linux}

\begin{wrapfigure}
  {r}{.125\textwidth}
  \centering
  \includegraphics[width=.125\textwidth]{images/logos/alpine.png}
\end{wrapfigure}

Alpine Linux\footnote{\url{https://www.alpinelinux.org}} is a lightweight, security-focused,
and resource-efficient distribution. In recent years, has grown in popularity as
the foundation for the majority of Docker images. It is based on musl\footnote{\url{https://musl.libc.org}}
(an implementation of the C standard library, an alternative to glibc), BusyBox\footnote{\url{https://busybox.net}}
(a single, small executable that combines tiny versions of many common UNIX utilities),
a custom package manager called APK\footnote{\url{https://wiki.alpinelinux.org/wiki/Alpine_Package_Keeper}}
and the OpenRC init system ()\cite{alpine_linux}. \\ % TODO OpenRC reference
Alpine Linux is the preferred reCluster distribution. It has been utilized and
tested during development and is now present in every cluster node.

\subsubsection{Arch Linux}
\label{subsubsec:implementation_distributions_iso_arch_linux}

\begin{wrapfigure}
  {r}{.125\textwidth}
  \centering
  \includegraphics[width=.125\textwidth]{images/logos/arch.png}
\end{wrapfigure}

Arch Linux\footnote{\url{https://archlinux.org}} is a general-purpose
distribution that focuses on simplicity, minimalism, and code elegance. It is based
on Systemd init system () % TODO Systemd reference
and strives to stay bleeding edge, offering the latest stable versions of most
software through Pacman\footnote{\url{https://wiki.archlinux.org/title/Pacman}}
package manager. Uses a rolling release system that allows one-time installation
and perpetual software upgrades, allowing to not reinstall or upgrade the system
from one version to the next. Notably, Arch Linux is the foundation for several
many popular enterprise-grade distributions, such as Manjaro\footnote{\url{https://manjaro.org}}
and SteamOS\footnote{\url{https://store.steampowered.com/steamos}},
demonstrating its customizability, power, and stability\cite{arch_linux}.

\section{Dependencies}
\label{sec:implementation_dependencies}

\subsection{Air-Gap Environment}
\label{subsec:implementation_dependencies_air_gap_environment}

\subsection{Management}
\label{subsec:implementation_dependencies_management}

\subsubsection{Configuration}
\label{subsec:implementation_dependencies_management_configuration}

\subsubsection{Script}
\label{subsec:implementation_dependencies_management_script}

\subsection{K3s}
\label{subsec:implementation_dependencies_k3s}

\begin{wrapfigure}
  {r}{.125\textwidth}
  \centering
  \includegraphics[width=.125\textwidth]{images/logos/k3s.png}
\end{wrapfigure}

\subsection{Node Exporter}
\label{subsec:implementation_dependencies_node_exporter}

\subsection{Prometheus}
\label{subsec:implementation_dependencies_prometheus}

\begin{wrapfigure}
  {r}{.125\textwidth}
  \centering
  \includegraphics[width=.125\textwidth]{images/logos/prometheus.png}
\end{wrapfigure}

\subsection{Autoscaler}
\label{subsec:implementation_dependencies_autoscaler}

\subsubsection{Cluster Autoscaler}
\label{subsec:implementation_dependencies_autoscaler_cluster_autoscaler}

\section{Installer}
\label{sec:implementation_installer}

\subsection{POSIX Shell}
\label{subsec:implementation_installer_posix_shell}