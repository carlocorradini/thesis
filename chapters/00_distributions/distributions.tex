\chapter{Distributions}
\label{cha:distributions}

Test

\section{Alpine Linux}
\label{sec:distributions_alpine_linux}

\begin{wrapfigure}
  {r}{.25\textwidth}
  \centering
  \includegraphics[width=.23\textwidth]{images/logos/alpine.png}
  \caption{Alpine Linux logo}
\end{wrapfigure}

Alpine Linux is a lightweight, security-focused, and resource-efficient
distribution. In recent years, has grown in popularity as the foundation for the
majority of Docker images. It is based on musl\footnote{\url{https://musl.libc.org}}
(an implementation of the C standard library, an alternative to glibc), BusyBox\footnote{\url{https://busybox.net}}
(a single, small executable that combines tiny versions of many common UNIX utilities),
a custom package manager called APK\footnote{\url{https://wiki.alpinelinux.org/wiki/Alpine_Package_Keeper}}
and the OpenRC init system ()\cite{alpine_linux}. % TODO OpenRC reference

\section{Arch Linux}
\label{sec:distributions_arch_linux}

% TODO Position
\begin{wrapfigure}
  {r}{.25\textwidth}
  \centering
  \includegraphics[width=.23\textwidth]{images/logos/arch.png}
  \caption{Arch Linux logo}
\end{wrapfigure}

Arch Linux is a general-purpose distribution that focuses on simplicity, minimalism,
and code elegance. It is based on Systemd init system () % TODO Systemd reference
and strives to stay bleeding edge, offering the latest stable versions of most
software through Pacman\footnote{\url{https://wiki.archlinux.org/title/Pacman}}
package manager. Uses a rolling release system that allows one-time installation
and perpetual software upgrades, allowing to not reinstall or upgrade the system
from one version to the next. Notably, Arch Linux is the foundation for several
many popular enterprise-grade distributions, such as Manjaro\footnote{\url{https://manjaro.org}}
and SteamOS\footnote{\url{https://store.steampowered.com/steamos}},
demonstrating its customizability, power, and stability\cite{arch_linux}.