\chapter{Distributions}
\label{cha:distributions}

Test

\section{Alpine Linux}
\label{sec:distributions_alpine_linux}

Alpine is a security-centered and very lightweight distro that is usually employed
to build firewalls, routers, VPN, VoIP, servers, and set-top boxes. Lately, it
has gained increasing popularity as a minimal Docker image (only 5MB). It's built
around musl (a C library alternative to glibc) and BusyBox (an alternative to
the several Linux core utilities in one executable), both very lightweight
alternatives. The security features include a Linux kernel with PaX and grsecurity
patches and stack-smashing protection (SSP) in all of the packages. Also, Alpine
can be loaded and run from memory RAM, one of the reasons why it is used in
embedded devices. Alpine use its own package manager, APK, with a considerable package
base given its lightweight nature\cite{alpine_linux}.

\section{Arch Linux}
\label{sec:distributions_arch_linux}

Arch Linux is an independently developed, x86-64 general purpose GNU/Linux
distribution versatile enough to suit any role. Development focuses on simplicity,
minimalism, and code elegance. Arch is installed as a minimal base system, configured
by the user upon which their own ideal environment is assembled by installing only
what is required or desired for their unique purposes. (...) Arch strives to
stay bleeding edge, and typically offers the latest stable versions of most software.
Arch Linux uses its own Pacman package manager, which couples simple binary packages
with an easy-to-use package build system. (...) Arch Linux uses a "rolling release"
system which allows one-time installation and perpetual software upgrades. It is
not generally necessary to reinstall or upgrade your Arch Linux system from one "version"
to the next. By issuing one command, an Arch system is kept up-to-date and on the
bleeding edge. Arch strives to keep its packages as close to the original
upstream software as possible\cite{arch_linux}.