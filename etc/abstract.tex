\chapter*{Abstract}
\label{abtract}
\addcontentsline{toc}{chapter}{Abstract}


Many large-scale computing clusters employed to host online services are mainly
concerned with maximizing overall performance and responsiveness, with the energy
impact being considered only for cost savings, while overall sustainability is
mostly marginal. Additionally, commercial data centers are largely homogeneous
relying on modern and cutting-edge hardware components. If we want data centers to
become more sustainable, we have to rethink their architecture and include
awareness of the energy and resources involved. In this work, a possible
solution for integrating and making sustainability as essential as performance
and responsiveness in data centers is proposed. As a direct consequence, this
new cluster architecture must preserve as much of the previous architecture's
usability and overall requirements as possible, while also being more aware of the
various resources and minimizing its energy consumption.

This thesis presents a case study of a possible heterogeneous data center architecture
that is more resource-aware, and generally sustainable. We guide the reader through
the design process, starting from guiding principles, progressing through theory
and, finally, implementation and critical reflections.

We open the thesis by outlining the context, goals, and core principles around which
the cluster is established. A dynamic and somewhat abstract context that is far
from static and should adapt/transform to the needs and requirements of the
organization managing the cluster. A set of goals aimed at making the cluster appear
less heterogeneous and more like a single, homogeneous entity. A collection of core
principles that enhances and provides a basis for improved sustainability, while
minimizing waste as much as possible: Sustainability, Acknowledging Planetary
Limits, Hardware Reusability, Energy Reduction, Insourcing, Interoperability, Free/Libre
And Open Source Software, and Dependencies Reduction.

After the foundations, we depicts the theoretical architecture, which defines a
high-level overview of the whole cluster. This is a collection of all the involved
hardware and software components, along with their respective responsibilities
outlining how they manage and maintain the cluster continuously and with an
awareness of energy and resources. To communicate, the components must be
connected, providing a network that enables workload orchestration and cluster management.

As the core point of the thesis, we detailed how we implemented a prototype
heterogeneous cluster that integrates the core principles with the theorized
networks and components. We named it reCluster, and the prefix "re" refers to
the Right to Repair movement's "reduce, reuse, repair, recycle" slogan. Among
the most notable reCluster components are the Server, which is a completely new implementation
developed from the ground up, and the Cluster Autoscaler, which is a reimplementation
that allows its core to communicate with the Server and vice versa. Combining
Server and Cluster Autoscaler allows the cluster architecture to perform
autoscaling procedures like upscaling and downscaling automatically. Moreover,
we developed a script that automates all of the necessary installation, configuration,
and deployment procedures. The prototype provides a totally autonomous, resource-aware,
and sustainable cluster by employing only those resources that are strictly necessary,
preferring those that are more sustainable than others. \\ %

The initial reCluster concept was theorized in \cite{conceptualising_resource_aware},
where I contributed to the Kubernetes analysis and the conceptualization of the
various algorithms, which were eventually implemented in the prototype. \\ %

Adapting what is already available and specifically developed for large and
energy-hungry data centers to a more sustainable and resource-aware architecture
while preserving the same original computing industry requirements is not only
possible, but also necessary for a more sustainable future.