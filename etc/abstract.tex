\chapter*{Abstract}
\label{abtract}
\addcontentsline{toc}{chapter}{Abstract}


Existing cluster architectures are primarily concerned with maximizing overall
performance and responsiveness, without consideration for the overall energy impact
and the many resources involved. Additionally, they are mostly composed of homogenous,
generally virtualized, machines that rely on modern and cutting-edge hardware components.
As a result, there is a need to reconsider the various methodologies and principles
involved, since it is becoming increasingly critical --- nearly unavoidable ---
to develop or transform cluster architectures orientated toward resource-awareness
and sustainability. The latter two requirements, where one does not negatively
influence the other, must be deemed as critical and significant as performance and
responsiveness. As a direct consequence, this new cluster architecture must retain
as much as possible of the previous architecture's usability and overall requirements
while being more mindful of the various components involved and its overall energy
consumption, achieving a more sustainable architecture not only suitable for the
present, but most importantly, for the future that is yet to come.

This document attempts to outline the many principles and methodologies that
should be considered when designing and developing a more resource-aware, sustainable,
and heterogeneous cluster architecture. The layout is intended to guide the reader
through the different procedures, decisions, and techniques that led to a potential
sustainable cluster implementation. Every preceding chapter/section provides the
theoretical and practical foundation for the subsequent ones.

Before defining any kind of architecture or implementation, it is important to outline
the context in addition to the core principles around which the cluster is established.
A dynamic and somewhat abstract context that is far from static and should adapt/transform
to the needs and requirements of the organization managing the cluster. A collection
of core principles that enhances and provides a basis for improved sustainability,
reusability, interoperability, and resource awareness while minimizing waste as
much as possible.

After the foundations, the theoretical architecture is depicted, which defines a
high-level overview of the whole cluster. This is a collection of all the involved
hardware and software components, along with their respective responsibilities
outlining how they manage and maintain the cluster continuously and with an
awareness of energy and resources. To communicate, the components must be
connected, providing a network that enables workload orchestration and cluster management.
Three network layers have been identified, each of which depicts a specific specialization:
K8s, Internal, and External; where the first layer is contained to the next layer,
and so on.

Finally, to develop a more resource-aware and sustainable architecture, an
operational heterogeneous prototype cluster that integrates the core principles
with the theorized networks and components has been implemented. The latter is identified
by the name reCluster, which is derived from the combination of the words
recycling (since only reconditioned hardware is being used) and cluster. The implementation
defines the relationship between the conceptual components and their software counterparts.
Some of these applications are partially pre-existing, but the two most important,
the Server and Cluster Autoscaler, necessitate, respectively, a fully custom implementation
built from the ground up and a custom Cloud Provider implementation that enables
the Cluster Autoscaler's core to communicate with the Server and vice versa.
Combining Server and Cluster Autoscaler allows the cluster architecture to
perform autoscaling procedures like upscaling and downscaling automatically. The
latter provides a completely autonomous, resource-aware, and sustainable cluster
by employing only the resources that are strictly required, preferring those
that are more sustainable than others.